\chapter{Modeling for Observational Treatment Comparisons}
\section{Propensity Score}
\bi
\item In observational studies comparing treatments, need to adjust
  for nonrandom treatment selection
\item Number of confounding variables can be quite large
\item May be too large to adjust for them using multiple regression,
  due to overfitting (may have more potential confounders than 
  outcome events)
\item Assume that all factors related to treatment selection that are
  prognostic are collected
\item Use them in a flexible regression model to predict treatment
  actually received (e.g., logistic model allowing nonlinear effects)
\item \textbf{Propensity score} (PS) = estimated probability of getting
  treatment B vs.\ treatment A
\item Use of the PS allows one to aggressively
  adjust for confounders by simulating a randomized trial if all
  confounders are in the PS model
\item Doing an adjusted analysis where the adjustment variable is the
  PS simultaneously adjusts for all the variables in the
  score \emph{insofar} as confounding is concerned (but \textbf{not
    with regard to outcome heterogeneity})
\item If after adjusting for the score there were a residual imbalance
  for one of the variables, that would imply that the variable was
  not correctly modeled in the PS
\item E.g.: after holding PS constant there are more subjects above
  age 70 in treatment B; means that age$>70$ is still predictive of
  treatment received after adjusting for PS, or age$>70$ was not
  modeled correctly.
\item An additive (in the logit) model where all continuous baseline
  variables are splined will result in adequate adjustment in the
  majority of cases---certainly better than categorization.  Lack of
  fit will then come only from omitted interaction effects.  E.g.: if
  older males are much more likely to receive treatment B than
  treatment A than what would be expected from the effects of age and
  sex alone, adjustment for the additive propensity would not
  adequately balance for age and sex. 
\ei

\section{Assessing Treatment Effect}
\bi
\item Eliminate patients in intervals of PS where there is no overlap
  between A and B, or include an interaction between treatment and a
  baseline characteristic\footnote{To quote Gelman and
    Hill~\cite[Section 10.3]{gel06dat}, ``Ultimately, one good
    solution may be a multilevel model that includes treatment
    interactions so that inferences explicitly recognize the decrease
    precision that can be obtained outside the region of overlap.''}
\item Many researchers stratify the PS into quintiles, get treatment
  differences within the quintiles, and average these to get
  adjustment treatment effects
\item Often results in imbalances in outer quintiles due to skewed
  distributions of PS there
\item Can do a matched pairs analysis but depends on matching
  tolerance and many patients will be discarded when their case has
  already been matched
\item Usually better to adjust for PS in a regression model
\item Model: $Y = \textrm{treat} + \log\frac{PS}{1-PS} +$
  nonlinear functions of $\log\frac{PS}{1-PS} +$ important prognostic
    variables
\item Prognostic variables need to be in outcome ($Y$) model even
  though they are also in the PS, to account for subject outcome heterogeneity
  (susceptibility bias)
\item If outcome is binary and can afford to ignore prognostic
  variables, use nonparametric regression to relate PS to outcome
  separately in actual treatment A vs.\ B groups
\item Plotting these two curves with PS on $x$-axis and looking at
  vertical distances between curves is an excellent
  way to adjust for PS continuously without assuming a model
\ei

\subsection{Problems with Propensity Score Matching}
\bi
\item The choice of the matching algorithm is not principle-based so
  is mainly arbitrary.  Most matching algorithms are dependent on the
  order of observations in the dataset.  Arbitrariness of matching
  algorithms creates a type of non-reproducibility. 
\item Non-matched observations are discarded, resulting in a loss of
  precision and power.
\item Matching not only discards hard-to-match observations (thus
  helping the analyst correctly concentrate on the propensity overlap
  region) but also discards many ``good'' matches in the overlap
  region.
\item Matching does not do effective interpolation on the interior of
  the overlap region.
\item The choice of the main analysis when matching is used is not
  well worked out in the statistics literature.  Most analysts just
  ignore the matching during the outcome analysis.
\item Even with matching one must use covariate adjustment for strong
  prognostic factors to get the right treatment effects, due to
  non-collapsibility of odds and hazards ratios.
\item Matching hides interactions with treatment and covariates.
\ei
Most users of propensity score matching do not even entertain the
notion that the treatment effect may interact with propensity to
treat, must less entertain the thought of individual patient
characteristics interacting with treatment.

\section{Recommended Statistical Analysis Plan}
\be
\item Be very liberal in selecting a large list of potential
  confounder variables that are measured pre-treatment.
\item If the number of potential confounders is not large in
  comparison with the effective sample size, use direct covariate
  adjustment instead of propensity score adjustment.  For example, if
  the outcome is binary and you have more than 5 events per covariate,
  full covariate adjustment probably works OK.
\item Model the probability of receiving treatment using a flexible
  statistical model that makes minimal assumptions (e.g., rich
  additive model that assumes smooth predictor effects).  If there are
  more than two treatments, you will need as many propensity scores as
  there are treatments, less one, and all of the logic propensity
  scores will need to be adjusted for in what follows.
\item Examine the distribution of estimated propensity score
  separately for the treatment groups. 
\item If there is a non-overlap region of the two distributions, and
  you don't want to use a more conservative interaction analysis (see
  below), exclude those subjects from the analysis.  Recursive
  partitioning can be used to predict membership in the non-overlap
  region from baseline characteristics so that the research findings
  with regard to applicability/generalizability can be better
  understood. 
\item Use covariate adjustment for propensity score for subjects in
  the overlap region.  Expand logit propensity using a restricted
  cubic spline so as to not assume linearity in the logit in relating
  propensity to outcome. Also include pre-specified important
  prognostic factors in the model to account for the majority of
  outcome heterogeneity.  It is not a problem that these prognostic
  variables are also in the propensity score.
\item As a secondary analysis use a chunk test to assess whether there
  is an interaction with logit propensity to treat and actual treatment.
  For example, one may find that physicians are correctly judging that
  one subset of patients should usually be treated a certain way.
\item Instead of removing subjects outside the overlap region, you
  could allow
  propensity or individual predictors to interact with treatment.
  Treatment effect estimates in the presence of interactions are
  self-penalizing for not having sufficient overlap.  Suppose for
  example that age were the only adjustment covariate and a propensity
  score was not needed.  Suppose that for those with age less than 70
  there were sufficiently many subjects from either treatment for
  every interval of age but that when age exceeded 70 there were only
  5 subjects on treatment B.  Including an age $\times$ treatment
  interaction in the model and obtaining the estimated outcome difference 
  for treatment A vs.\ treatment B as a function of age will have
  a confidence band with minimum width at the mean age, and
  above age 70 the confidence band will be very wide.  This is to be
  expected and is an honest way to report what we know about the
  treatment effect adjusted for age.   If there were no age $\times$
  treatment interaction, omitting the interaction term would yield a
  proper model with a relatively narrow confidence interval, and if
  the shape of the age relationship were correctly specified the
  treatment effect estimate would be valid.  So one can say that not
  having comparable subjects on both treatments for some intervals of
  covariates means that either (1) inference should be restricted to
  the overlap region, or (2) the inference is based on model assumptions.
  \ee

Using a full regression analysis allows interactions to be explored,
as briefly described above.
Suppose that one uses a restricted cubic spline in the logit
propensity to adjust for confounding, and all these spline terms are
multiplied by the indicator variable for getting a certain treatment.
One can make a plot with predicted outcome on the $y$-axis and PS on
the $x$-axis, with one curve per treatment.  This allows inspection of
parallelism (which can easily be formally tested with the chunk test)
and whether there is a very high or very low PS region where treatment
effects are different from the average effect.  For example, if
physicians have a very high probability of always selecting a certain
treatment for patients that actually get the most benefit from the
treatment, this will be apparent from the plot.

\section{Reasons for Failure of Propensity Analysis}
Propensity analysis may not sufficiently adjust for confounding in nonrandomized studies when
\bi
\item prognostic factors that are confounders are not measured and are not highly correlated with factors that are measured
\item the propensity modeling was too parsimoneous (e.g., if the researchers excluded baseline variables just because they were insignificant)
\item the propensity model assumed linearity of effects when some were really nonlinear (this would cause an imbalance in something other than the mean to not be handled)
\item the propensity model should have had important interaction terms that were not included (e.g., if there is only an age imbalance in males)
\item the researchers attempted to extrapolate beyond ranges of overlap in propensity scores in the two groups (this happens with covariate adjustment sometimes, but can happen with quantile stratification if outer quantiles are very imbalanced)
\ei

\section{Sensitivity Analysis}
\bi
\item For $n$ patients in the analysis, generate $n$ random values of
  a hypothetical unmeasured confounder $U$
\item Constrain $U$ so that the effect of $U$ on the response $Y$ is
  given by an adjusted odds ratio of $OR_{Y}$ and so that $U$'s distribution is
  unbalanced in group A vs.\ B to the tune of an odds ratio of
  $OR_{treat}$.
\item Solve for how large $OR_{Y}$ and $OR_{treat}$ must be before the
  adjusted treatment effect reverses sign or changes in statistical
  significance
\item The larger are $OR_Y$ and $OR_{treat}$ the less plausible it is
  that such an unmeasured confounder exists
\ei
See the \R\ \co{rms} package \co{sensuc} function.

\section{Reasons To Not Use Propensity Analysis}
Chen et al.~\cite{che16too} demonstrated advantages of using a unified
regression model to adjust for ``too many'' predictors by using
penalized maximum likelihood estimation, where the exposure variable
coefficients are not penalized but all the adjustment variable
coefficients have a quandratic (ridge) penalty.

\section{Further Reading}
\href{http://www.stat.columbia.edu/~gelman/arm/chap10.pdf}{Gelman} has
a nice chapter on causal inference and matching from Gelman and
Hill~\cite{gel06dat}. 
